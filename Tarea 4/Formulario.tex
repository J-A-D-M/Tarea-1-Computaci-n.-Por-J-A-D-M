\documentclass[12pt,letterpaper]{article}
\usepackage[utf8]{inputenc}
\usepackage[spanish]{babel}
\usepackage[rmargin=2.5cm,lmargin=3cm,tmargin=3cm,bottom=3cm]{geometry}

\usepackage{fancyhdr}
    \pagestyle{fancy}
        \fancyhf{}
        \rhead{\textcolor{white}{\thepage}}
        \cfoot{\textcolor{white}{Jordán Aarón Duarte Martínez}}
        \lhead{\textcolor{white}{\textsc{MATEMÁTICAS}}}

\usepackage[x11names]{xcolor}
\usepackage{graphicx}
\usepackage{caption}
\usepackage[rightcaption]{sidecap}

\usepackage{amsmath}
\usepackage{amssymb}
\usepackage{dsfont}
\usepackage{latexsym}
\usepackage{mathrsfs}

\parindent=0mm


\begin{document}
\pagecolor{black}
\color{white}

\begin{center}
    {\textcolor{Goldenrod1}{\textbf{\textsc{\underline{\Huge{1 FORMULARIO}}}}}}
\end{center}

\begin{center}
    {\textcolor{Goldenrod1}{\textbf{\textsc{\underline{\Huge{PARA VIAJES TEMPORALES}}}}}}
\end{center}

\begin{center}
    {\textcolor{Goldenrod1}{\textbf{\textsc{\large{14/10/40000}}}}}
\end{center}

\begin{center}
    {\textcolor{Goldenrod1}{\textsc{UNIDAD DE COMANDO ESPACIAL}}}
\end{center}

Como nuevo miembro de la subdivisión de viajes temporales, 
se puede dar el caso de que alguno de nuestros agentes (tú) se quede barado en algún punto de la linea temporal. Debido a ello se crearon una serie de manuales para controlar aquella situaciones, y proceder sin afectar negativamente nuestro tiempo actual de civilización intergaláctica.\newline

No obstante, este 1 formulario que forma parte del manual ``¿Qué hacer si ya destruí toda posibilidad de volver la línea temporal a su originalidad?" debe ser tomado como última medida por obvias razones; sin embargo, confiamos en ti, agente, para que en este caso seas capaz de usar tus conocimientos básicos del entrenamiento en conjunto con el manual, y este formulario, para llevar a la humanidad antigua a un punto tan, o más, alto que nuestra sociedad.\newline

Buena suerte, agente. A partir de ahora estás sólo.\newline

\section*{MATEMÁTICAS}

%|||||||||||||||||||||INICIO_DE_LISTADO|||||||||||||||||||||
\begin{itemize}

    \item[\#] \textbf{Teorema de Pitágoras.}\newline
    
    Formulada en el año 530 antes de Cristo por Pitágoras, en ella se describe la relación entre los lados de un triángulo rectángulo en una superficie plana tal que el cuadrado del lado más largo (hipotenusa [c]) es igual a la suma de los cuadrados de los otros dos lados (catetos a,b). Conceptos esenciales para la comprensión de la geometría.
    
    $$a^{2}+b^{2}=c^{2}$$
    
    \item[] \textbf{Logaritmos (y sus propiedades).}\newline

        \begin{itemize}
            
            \item[\&] \textit{Definición del logaritmo.}\newline
            
            Descritos en 1610 por John Napier. Su definición básica sería que: dado un número real (x), la función logaritmo le asigna un exponente real (n) a un número real positivo fijo (b), llamado base, tal que b elevado a la n da x. De esta noción básica se deduce que es la función inversa de b a la potencia n.
    
            $$\log _{b}x=n \Leftrightarrow b^{n}=x$$
            
            \item[\%] \textit{Propiedad multiplicativa.}\newline
    
            El logaritmo de un producto (xy) es igual a la suma de los logaritmos de los factores.
            
            $$\log _{b}(x\cdot y)=\log _{b}(x)+\log _{b}(y)$$
            
            \item[\$] \textit{Propiedad fraccionaria.}\newline
    
            El logaritmo de un cociente (x sobre y) es igual al logaritmo del numerador menos el logaritmo del denominador.
            
            $$\log_{b} (\frac{x}{y}) = \log_{b}(x) -\log_{b}(y)$$
            
            \item[$\aleph$] \textit{Propiedad exponencial.}\newline
    
            El logaritmo de una potencia ($x^{y}$) es igual al producto entre el exponente y el logaritmo de la base de la potencia.
            
            $$\log_{b}(x^{y})=y\log_{b}(x)$$
            
        \end{itemize}

    Gracias a estas formulas los ordenadores pueden simplificar operaciones muy complejas, y así realizar cálculos exactos y rápidos de multiplicaciones de grandes cantidades.\newline

    \item[$\angle$] \textbf{Definición de la derivada.}\newline
    
    Descrita por Isaac Newton en 1668. A la base de la definición se le entiende como la función igual a la diferencia de un intervalo, dividida por el aumento del intervalo inicial, perimitiendo así que al momento de obtener el límite cuando tiende a 0, se dé como resultado otra función, denomidada función derivada, que describe los más mínimos cambios del intervalo de la función inicial.\newline

    \begin{equation*}
    \textcolor{Magenta1}{\frac{df}{dt}=\lim_{h\rightarrow 0}\frac{f(t+h)-f(t)}{h}}
    \end{equation*}\newline

    Ésta ecuación ayudó a comprender el cambio de las funciones cuando sus variables cambiaban, permitiendo describir mucho mejor una multitud enorme de fenomenos de todo tipo.\newline
    
    \item[$\backprime$] \textbf{Definición del número imaginario ``i".}\newline
    
    Descritos en conjunto por Johann Carl Friedrich Gauss, y Leonhard Euler en 1750, el número imaginario ``i" se comprende como la raíz de menos uno (y por tanto ``i" al cuadrado es igual a -1); siendo llamado, precisamente, imaginario por el hecho de que no existe ningun número real que elevado al cuadrado te dé menos uno.

    $$i^{2}=-1$$

    Gracias a esta definición se dió lugar a los números complejos, esenciales para resolver muchos problemas.\newline
    
    \item[] \textbf{Teorema de los poliedros.}\newline
    
    Descrito por Euler en 1751, explica la relación del núm. de caras (C), núm. de vértices (V), núm. de aristas (A), núm. de lados (n), núm. de aristas que convergen en los vértices (r) de poligonos regulares tridimenscionales; siendo estas relacionaes tal que así:

        \begin{itemize}
        
            \item[$\backslash$] \textit{Característica de Euler (1 Forumla de los poliedros).}\newline

            El número de vértices más el número de caras de un poliedro es igual al número de aristas más dos.

            $$V+C=A+2$$
            
            \item[$\bigstar$] \textit{2 Formula de los poliedros.}\newline

            El producto del número de lados por el número de caras de un poliedro es igual a dos veces el número de aristas.
            
            $$n\cdot C=2A$$

            \item[$\blacklozenge$] \textit{3 Formula de los poliedros.}\newline

            El proucto del número de de aristas, que convergen en los vertices, por el número de vertices es igual a dos veces el número de aristas.
            
            $$r\cdot V=2A$$
            
            \item[$\blacksquare$] \textit{4 Formula de los poliedros.}\newline

            La suma de: la división de dos veces el número de aristas sobre el número de de aristas que convergen en los vertices; menos el número de aristas; más la división de dos veces el número de aristas sobre el número de lados es igual a dos.
            
            $$\frac{2A}{r}-A+\frac{2A}{n}=2$$
            
            \item[$\blacktriangle$] \textit{5 Formula de los poliedros.}\newline

            La suma de los inversos del número de lados con el número de número de aristas que convergen en los vertices es igual a un medio más el inverso del número de aristas.
            
            $$\frac{1}{n}+\frac{1}{r}=\frac{1}{2}+\frac{1}{A}$$
            
        \end{itemize}

    Además de ser un gran aporte a la geometría, con estas formulas surgiría la topología.\newline
    
    \item[$\blacktriangledown$] \textbf{La Distribución normal.}\newline
    
    Una ecuación empleada en una gran variedad de ciencias que usan la estadística. La ecuación, supuestamente, fue formulada en 1810 por Carl Friedrich Gauss, aunque hay cierta polémica sobre si su verdadera atribución corresponde más bien a Abraham de Moivre, en 1733, y extendido por Laplace en 1812. De cualquier modo la formula establece las siguientes formas para entenderla:
    
    $$\Phi _{\mu ,\sigma ^{2}}(x)=\int \limits_{-\infty}^{x} \phi _{\mu ,\sigma ^{2}}(u)du=\frac{1}{\sigma \sqrt{2\pi }}\int \limits_{-\infty}^{x}e^{-\frac{{u-\mu}^{2}}{2\sigma ^{2}}}du, \quad x\in \mathds{R}$$

    Donde: $\mu$ es la media (también puede ser la mediana, la moda o el valor esperado, según aplique), $\sigma$ es la desviación típica, $\sigma^{2}$ es la varianza, y $\phi$  representa la función de densidad de probabilidad.\newline

    \item[$\bot$] \textbf{La transformada de Fourier.}\newline
    
    Jean-Baptiste Joseph Fourier la formuló en 1822. Donde se relacionana dos funciones, f y g, tal que f es $L^{1}$, es decir, f tiene que ser una función integrable en el sentido de la integral de Lebesgue; mientras que el factor fraccionario multiplicado por la integran facilita el enunciado de algunos de los teoremas referentes a la transformada de Fourier.

    $$g(\xi )=\frac{1}{\sqrt{2\pi }}\int_{-\infty }^{+\infty }f(x)e^{-i\xi x}dx$$

    Sus aplicaciones son muchas en las áreas de la matemática, física, ingeniería, la teoría de los números, la combinatoria, el procesamiento de señales (electrónica), la teoría de la probabilidad, la estadística, la óptica, la propagación de ondas y otras áreas.\newline

    \item[$\Box$] \textbf{Teoría de la información.}\newline
    
    Mide el contenido de información de un mensaje y describe el límite hasta el que se puede comprimir la información. El responsable de esta ecuación fue Claude Elwood Shannon y la fórmula data de 1949.
    
    $$H=\sum_{i=0}^{n}P_{i}L_{i}, \quad H=\sum_{i=0}^{n}-P_{i}\log_{2}P_{i}$$

    Donde: Pi es la probabilidad de ocurrencia del mensaje-i de una fuente, Li es la longitud del código utilizado para representar a dicho mensaje, y H es ``la Entropía de la fuente" que determina el nivel de compresión que se puede obtener como máximo para un conjunto de datos.\newline

    Sin embargo, haciendo unas cuantas modificaciones ehn la primer ecuación podemos notar que se puede obtener a H como la longitud mínima con la cual puede codificarse un mensaje al tomar a $L_{i}=\log_{2}(1/P_{i})=-log_{2}(P_{i})$, con lo que la formula quedaría como la de la derecha.

\end{itemize}
%|||||||||||||||||||||||FIN_DE_LISTADO||||||||||||||||||||||
\newpage

    \pagestyle{fancy}
            \fancyhf{}
            \rhead{\textcolor{white}{\thepage}}
            \cfoot{\textcolor{white}{Jordán Aarón Duarte Martínez}}
                 \lhead{\textcolor{white}{\textsc{FÍSICA}}}

\section*{FÍSICA}

%|||||||||||||||||||||INICIO_DE_LISTADO|||||||||||||||||||||
\begin{itemize}

    \item[$\Re$] \textbf{Ley de la inversa del cuadrado de la gravitación universal.}\newline
    
    Formulada en 1687 por Isaac Newton, esta ecuación explica que la fuerza de atracción gravitacional ($\vec{F}$) es igual al producto de la constante de gravitación universal ($G=6.67384\times 10^{-11}$) por la división del producto de las masas ($m_{1},m_{2}$) sobre la distancia al cuadrado ($\vec{r}^{2}$).
    
    $$\vec{F}=G\frac{m_{1}\cdot m_{2}}{\vec{r}^{2}}$$

    Ésta ayudó a comprender el funcionamiento de la gravedad a nivel universal, unificando en una sola ecuación fenómenos aparentemente tan diferentes como la caída de una manzana y las órbitas de los planetas.\newline
    
    \item[$\clubsuit$] \textbf{Ecuación de d'Alembert.}\newline
    
    Jean le Rond d'Alembert la propusó en 1747. Es sino una ecuación diferencial que describe cómo una propiedad está cambiando a través del tiempo en términos del derivado de esa propiedad, es decir, describe la propagación de una variedad de ondas.

    $$\frac{\partial u}{\partial t^{2}}=c^{2}\frac{\partial u}{\partial x^{2}}$$

    Con esta formula, y sus multiples modificaciones, se pueden describir a las ondas sonoras, las ondas de luz, las ondas en el agua, y una multitud de ondas que no parecen en nada relacionadas.\newline
    
    \item[$\dashv$] \textbf{Ecuaciones de Navier-Stokes.}\newline
    
    Claude-Louis Henri Navier y George Gabriel Stokes describieron esta ecuación, en 1845, para explicar la mecánica de fluidos. Consisten en una ecuación de continuidad dependiente del tiempo para la conservación de la masa, tres ecuaciones de conservación del momento dependientes del tiempo, y una ecuación de conservación de la energía dependiente del tiempo, i.e: cuatro variables independientes (x,y,z,t).

    $$\rho \frac{D\vec{V}}{Dt}=-\bigtriangledown  p+\rho \vec{g}+\mu \bigtriangledown^{2}V^{2}$$

    Esta ecuación y sus multiples soluciones son literalmente la base de la aerodinámica y la hidrodinámica.\newline
    
    \item[] \textbf{Ecuaciones de Maxwell.}\newline
    
    Originalmente 20 ecuaciones, James Clerk Maxwell reunió todas estas ecuaciones en 4, en 1863, trás largos años de resultados experimentales de Coulomb, Gauss, Ampere, Faraday y otros, lo que al final dió introducción a los conceptos de campo y corriente de desplazamiento, unificando los campos eléctricos y magnéticos en un solo concepto: el campo electromagnético, descrito por las siguientes ecuaciones, donde: la permitividad del vacío es dada por $\epsilon_{0}=8.854\times 10^{-12} F$, la permeablididad magnética es dada por $\mu_{0}=4\pi \times 10^{-7} H$, la cantidad de campo electrico dada por $\vec{E}$, la densidad de carga en el medio interior a la superficie cerrada dada por $\rho$, la densidad de flujo magnético dada por $\vec{B}$, y la densidad de corriente es dada por $\vec{J}$.\newline

        \begin{itemize}
        
            \item[$\Diamond$] \textit{Ley de Gauss.}\newline

            La forma diferencial de la ley de Gauss, en forma local, afirma que por el teorema de Gauss-Ostrogradsky, la divergencia del campo eléctrico es proporcional a la densidad de carga eléctrica, lo que venría a ser:
            
            $$\vec{\bigtriangledown} \cdot \vec{E}=\frac{\rho}{\epsilon_{0}}$$
            
            \item[$\diamondsuit$] \textit{Ley de Gauss para el campo magnético.}\newline

            Al encerrar un dipolo en una superficie cerrada, no sale ni entra flujo magnético, por lo tanto el campo magnético no diverge, no sale de la superficie. Entonces la divergencia es cero, o sea:
            
            $$\vec{\bigtriangledown} \cdot \vec{B}=0$$
            
            \item[$\ell$] \textit{Ley de Faraday.}\newline

            El rotacional del campo eléctrico es la derivada parcial de la inducción magnética con respecto al tiempo, i.e:
            
            $$\vec{\bigtriangledown} \times \vec{E}=-\frac{\partial \vec{B}}{\partial t}$$
            
            \item[$\emptyset$] \textit{Ley de Ampère generalizada.}\newline

            Si se tiene un conductor, un alambre recto por el que circula una densidad de corriente J, esta provoca la aparición de un campo magnético B rotacional alrededor del alambre y que el rotor de B apunta en el mismo sentido que J. Que en forma de ecuacón es:
            
            $$\vec{\bigtriangledown} \times \vec{B}=\mu_{0}\vec{J}+\mu_{0}\epsilon_{0}\frac{\partial \vec{E}}{\partial t}$$
            
        \end{itemize}
    
    \item[$\heartsuit$] \textbf{Segunda ley de la termodinámica.}\newline
    
    Formulada por Ludwig Boltzmann en 1874. Esta ecuación indica que, en un sistema cerrado, la entropía (S) es siempre constante o creciente a la transformación de calor (Q) producidad dividida por la temperatura de equilibrio del sistema (T); entendiendo a la entropia como la tendencia al orden, o bien, de un modo más formal, como el número de microestados compatibles con el macroestado de equilibrio del sistema.
    
    $$dS\geq \frac{\delta Q}{T}$$

    Se trata de una de las leyes más importantes de la física por muchas razones, pero una de ellas es que implica que ``la cantidad de entropía del universo tiende a incrementarse en el tiempo”, y, por lo tanto, el universo en algún momento muy lejano estará en equilibrio térmico total, i.e: no habrá más cambios, no más estrellas, no más vida, y el tiempo dejará de tener sentido pues el universo entero dejará de tener cambios medibles durante el tiempo.\newline
    
    \item[$\mho$] \textbf{Teoría de la relatividad.}\newline
    
    Formulada en 1905 por Albert Einstein, esta conocidísima ecuación demuestra una increible relacione entre la masa y la energía, pues da a entender que la energía (E) es igual al producto de una masa en reposo (m) por la velocidad de la luz (c) al cuadrado; aunque, ha de añadirse, la forma extendida de la ecuación (la de hasta la izquierda) nos da más detalles gracias al valor ``p", que es el modulo del momento lineal de la masa, pues al igualarlo a 0 (masa en reposo) nos da la formula más conocida, mientras que al darle valores no cero obtenemos la ecuación para diversas masas relativistas.\newline

    \begin{equation*}
    \textcolor{lime}{E=mc^{2},\quad E=\sqrt{p^{2}c^{2}+m_{0}^{2}c^{4}}}
    \end{equation*}\newline
    
    \item[$\Join$] \textbf{Ecuación de Schrödinger.}\newline
    
    Formulada en 1927 por Erwin Schrödinger, describe la evolución temporal de una partícula subatómica masiva de naturaleza ondulatoria y no relativista. Así, el espacio no está vacío y cuando una partícula lo atraviesa, la deforma, y el espacio también genera una forma de onda por esta perturbación. La ecuación representa la probabilidad de que en un tiempo determinado se encuentre allí la partícula en las coodenadas X,Y y Z del espacio. En definitiva, describe la evolución de un sistema cuántico.

    \begin{equation*}
    \textcolor{Cyan1}{i\hbar \frac{\partial }{\partial t}\psi (r,t)=\hat{H}\psi (r,t)}
    \end{equation*}\newline

    Donde: i es la unidad imaginaria, $\hbar$ es la constante de Planck reducida, o constante de Dirac (constante de Planck dividida por $2\pi$), el símbolo $\frac{\partial }{\partial t}$ indica una derivada parcial con respecto al tiempo t, y $\psi$ es la función de onda del sistema cuántico, y $\hat{H}$ es el operador diferencial Hamiltoniano (el cual caracteriza la energía total de cualquier función de onda dada y tiene diferentes formas que dependen de la situación).
    
\end{itemize}
%|||||||||||||||||||||||FIN_DE_LISTADO||||||||||||||||||||||

%NOTA PAL PROFESOR: En la sección de física puse dos ecuaciones como las más importantes (a pesar de que se dijo que sólo fuera una), pero... ¿Cómo rayos elijo si es más importante la mécanica cuántica o la relatividad? XD. Bno, nomás es un detallito, pero espero que no cause problemas jsjsjs.

\end{document}