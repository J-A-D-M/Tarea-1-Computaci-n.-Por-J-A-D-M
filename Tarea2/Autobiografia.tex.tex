\documentclass[12pt,letterpaper]{article}
\usepackage[utf8]{inputenc}
\usepackage[T1]{fontenc}    
\usepackage[spanish]{babel}
\usepackage[top=2cm,bottom=2cm,left=3cm,right=3cm]{geometry}
\usepackage{xcolor}
\parindent=0mm
\pagestyle{myheadings}

\title{\underline{\textbf{Mamihlapinatapai}}}
\author{Jordán Aarón Duarte Martínez }
\date{11 de Septiembre del 2022}

\begin{document}

\maketitle

\section{\Huge{\textbf{Academia}}}
    \subsection{\huge{Pasado}}
{\texttt{Yendo en contra del orden cronológico estandar en el que las personas suelen hablar en sus Autobiografias, pues sale decir que vengo de la {\large{\textbf{Escuela Nacional Preparatoria Núm. 9 Pedro de Alba}}}, una escuela bastante gris en muchas de sus estructuras y colores presentes, aunque en estética lo salvaba un poco el pasto que se filtraba por entre los huecos del asfalto; también cabe señalar que en terminos de maestros era, con notabiliad, más variada que en su simple aspecto exterior, pues podías encontrar desde los maestros de como $70^{\infty^{\infty^{\infty^{\infty^{\infty}}}}}$ de años de edad que califican enviandote a examen final apenas entras, hasta los famosos profesores transatlánticos que sin problemas califican el periodo entero con un simple examen grupal, en el que encima te da 3 minutos antes las respuestas... Mas, haciendo un poco de memoria, también llegue a encontrar un cuasi-buen-profesor, es decir, un maestro que gusta de su materia, y que disfruta de enseñarla, aunque, para mi, y su, mala suerte, era bastante pasivo en cuanto a actitud, al punto en el que con las palabras adecuadas se pierde la clase entera en una conversación perpetua de un Alumno X con él.\\}}

{\texttt{Ahora, continuando con el famoso orden anti-cronológico, toca hablar de mi Escuela Secunadaria Núm. 15 Albert Einstein, lugar en el que, hasta hoy en día, he tenido mi etapa estudiantil más divertida, y de mejores calificaciones. En sí la llamaría como mi verdadera primaria en cuanto a actitud y algunos valores en lo que respecta a la escuela, pues en aquellos tiempos busque la excelencia académica como meta personal, pues durante años había sido de lo más bajo en la cadena estudiantil, tanto en calificaciones como en respeto, así que me planteé ir a ese escuela en específico porque sabía que casi nadie más de mi primari iría ahí, dandome paso libre a formarme una nueva personalidad con la que me sintiera más satisfecho en la mayoria de los aspectos... Ah, sí, y también era bastante bonita, arboles y plantas por doquier, además que estaba pintaba de muchos blancos y azules medio apagados, pero que hoy en día son nostálgicos.\\}}

{\texttt{Terminando con el "de regreso, mami" de explicación temporal, mi origen, que yo recuerde (estuve en un kinder, pero sólo recuerdo que una vez me disfrace de pollito rojo), fue aquel hermoso lugar de un monton de pastos, paredes de puro color rojo terracota, y pasillos enormes en los que siempre, en horas libres, había un niño dispuesto a jugar a algo con casi cualquiera... Definitivamente el peor lugar en el que estuve, y sólo por una razón:... yo. Por culpa de mi necesidad de ser aceptado socialmente dejaba que me hicieran menos y me bullearan. Ciertamente no era mala la primaria, pero esas simples acciones en la Escuela Primaria ANEXA a la Normal Luis Hidalgo Monroy provocaron que hasta hoy en día sea mi más largo e indeseado recuerdo de todos, al punto que realmente no recuerdo tanto de la primaria... y eso me alegra.\\}}

{\texttt{Ah... Cierto, se me olvidaba responder las otras dos preguntas que piden para la primera subsección, pues me imagino que se estará haciendo medio largo esto, ¿no? Bueno, supondré que la respuesta es una leve cabeceada de cabeza, a la vez que un ''sí'' resuena con el acento de la voz en tu cabeza que suena cuando lees. Por lo que lo explicaré rápido.\\}}

{\texttt{Mi Escuela preparatoria {\large{\textbf{quedaba relativamente cerca, pues solía tardar unos 45 minutos en llegar entrando en el metro ''Colegio militar'', transbordando en ''Hidalgo'', y unas cinco estaciones que hasta el metro ''Deportivo 18 de Marzo''}}}; mas, he de admitir, mi secundaria y primaria estaban mucho más cerca, literalmente podía salir de mi casa caminando, y en más o menos 10 minutos llegaba, incluso si caminaba tranquilo y sin prisa.}}\\

    \subsection{\huge{Actualidad}}

{\textsc{Hoy en día me encuentro estudiando la carrara de física en la UNAM, específicamente en la Facultad de ciencias, y ciertamente me gusta la escuela en todo su exterior, aunque por ahí he escuchado algunos ''recuerdos de Vietnam" , de los más veteranos que yo, que me hacen pensar si todo será miel sobre ojuelas (aunque en realiad casi nada es miel sobre ojuelas, sino es que literalemnte nada lo es... excepto la miel sobre ojuelas, claro). De cualquier modo, es una excelente escuela por lo que hasta ahora he visto, pues los profesores son muy buenos en terminos generales, además que {\large{\textbf{el traslado, en comparación a otros compañeros míos, es bastante corto, pues sólo tardo 45 minutos en ir y venir del metro ''Colegio militar'' hasta el metro ''Universidad" , y si el metro está de malas pues sería una hora.}}}}}

    \subsection{\huge{¿Por qué física?}}

{\Large{\textsl{¿Cómo que "Por qué''?\\}}}

{\textsl{Jejeje, bueno, ya hablando en serio. El porqué es tan simple como que {\large{\textbf{desde pequeño quería ser alguien que destacará por su ingenio}}}, así que en un principio parecía bastante claro sabiendo que los mayores y más famoso genios de la historia han sido físicos, pero con el tiempo me dí cuenta de que se podía demostrar genialidad en demasiados aspectos, y ahí fue cuando me pregunté que debería estudiar. Mi mente me decía que era obvio que física, después de todo es a lo que siempre habiamos aspirado, sin embargo, por aquellos tiempos empecé a descubrir nuevas aficiones en las que era especialmente bueno, por ejemplo: escribir, mas, después de unas cuantas meditaciones, terminé por concluir: "Bueno, una carrera en realidad no te obliga a trabajar de lo que estudiaste... sería lo ideal, claro, pero se dan tantos eventos en la vida, de los cuales sólo controlamos como reaccionaremos, que en realiad no importa demasiado aferrarse a un sueño que bien se podría convertir en una insalubre obseción". Y así termine por decidir que seguiría mi sueño, {\large{\textbf{pero que no estaría obligado a completarlo, y mucho menos a aferrarme contra viento y marea sólo porque es mi "propósito", cuando nunca ha existido un propósito, de hecho.}}}}}

\section{\Huge{\textbf{Hobbies o pasatiempos}}}

    \subsection{\huge{Escribir}}

{\textsf{El nombre del hobby es bastante descriptivo por sí mismo, empero no puedo irme de aquí sólo diciendo ''Es obvio, papaito", así que, en síntesis, es: {\large{\textbf{dar surgimiento a una idea, desarrollar esa idea}}}, fantasear con esa idea y sus cientos de posibilidades, meter personajes en esas ideas, meter situaciones peligrosas, meter un propósito, tener claro si matarás o no a X o Y personajes, y, finalmente, escribir como un desquiciado, y no poder avanzar porque a cada rato te vuelves a un anterior párrafo para reescribirlo y que quede ''Perfecto''.\\}}

{\textsf{En tiempos de {\large{\textbf{vacaciones sólia dedicarle más o menos hora y media diaria}}}, pero hoy día que la escuela está en su recien apogeo, pues como que no me ha dado tiempo (NOTA: esto no es una indirecta para que me deje menos tarea, profesor. Neta se lo juro... ¿Qué no me cree? Pues vera, se lo juro por mi santa mamaita, que no es santa, pero es una muy buena madre).}}

    \subsection{\huge{Dibujar}}

{\textsf{Tal y como el anterior Hobby, el nombre es bastante descriptivo, aunque supongo que podría haber muchos más detalles, después de todo hay muchas formas de dibujar, y muchas cosas que se pueden dibujar. Pero, para no alargar de más ésto, {\large{\textbf{sólo dibujo rostros (es lo que más me gusta, aunque no me niego a dibujar más cosas en un futuro)}}}, y generalmente en un estilo que es {\large{\textbf{entre anime y cartoon}}}, aunque suele tirar un poco más a cartoon (cosa rara considerando que últimamente veo más animes que caricaturas).\\}}

{\textsf{Me apena decir que nunca le he dedicado tant tiempo como me gustaría, sin embargo creo haber hecho algún que otro avance con el paso del tiempo, y haciendo, por lo general, varios dibujos rápidos de 10 minutos cada uno, con lo cual {\large{\textbf{en los buenos días suelo gastar unos 30 minutos, y en los malos unos 10 minutos antes de dormir.}}}}} 

\section{\Huge{\textbf{Géneros musicales favoritos}}}

{\textrm{Sin temor a equivocarme, sé que sería más sencillo explicar cuáles géneros musicales NO son mis favortios, después de todo casi todo género me gusta: algunas operas clásicas que recuerdo como suenan pero no me sé el nombre, una que otra música folclórica que no me sé el nombre, muchas de rock y sus subgeneros, pop y sus subgeneros, electrónica y muchos de sus subgeneros, una que otra poquita de hip hop, y algunas de reguetón antiguo (y el maldito gusto culposo de ''Despacito''). O sea, el único género que casi en su totalidad no me gusta (aunque estoy abierto a escucharlo en caso de que algún día saquen algo bueno para mí) es el Trap (Te estoy viendo a ti, Conejito malo).\\}}

{\textrm{Bueno, lo anterior sería mi respuesta sincera, sin embargo, para asegurarme de que no me bajen puntos diré los dos géneros que supongo que más destacan entre mis gustos, los cuales son: {\large{\textbf{rock y electrónica}}}, y si hablamos de sus subgéneros supongo que serían, respectivamente, Art rock (el de Bohemian Rhapsody) y Nightcore. Esos y muchos otros géneros suelo oirlos mientras voy de camino a la escuela, mientras estudio, y cuando escribo mis historias.\\}}

{\textrm{P.D. Se me olvidaba lo de los dos ejemplos de mis géneros favoritos, así que aquí están: {\large{\textbf{Bohemian Rhapsody, de Queen; y Nightcore - Hate me, de Eiden XII.}}}}}

\section{\Huge{\textbf{Explicación del porqué del título}}}

{\textit{No te lo esperabas, ¿ehhhhhhhh? Pues sí, toda esta sección consistirá en la explicación del porqué del título tan raro que elegí, el cual, por cierto, es una palabra del idioma extinto Yagán, y tiene el record guinness a la palabra más concisa del mundo, pues {\large{\textbf{su significado no es otro que: Una mirada entre dos personas, cada una de las cuales espera que la otra comience una acción que ambos desean pero que ninguno se anima a iniciar.\\}}}}}

{\textit{Una palabra perfecta para cuando dejas a tu cita en la puerta de su casa, y ambas se dan esa torpe mirada en la que desean besarse, pero ninguno sabe si debería animarse, o mejor esperar que el otro inicie. Aunque también aplicaría para esos casos en los que un amigo tuyo y tú está en una plática muy incomoda con un tercero, y ambos se ven diciendose casi telepaticamente:\\}}

{\small{\textit{-No mames, wey, este webón ya lleva más de hora y media hablandonos, y eso que teníamos dos horas libres antes de cálculo II.\\}}}

{\small{\textit{-Sí, ya sé, tenemos que enviarlo a la verdolaga, pero no podemos hacerlo así de fácil; ¿qué tal si se hace el ofendido y difunde mentrias sobre nosotros y nos funan?\\}}}

{\small{\textit{-Vale queso, eso sería si estuvieramos en la facultad de Filosofía y letras, aquí no hay tanto y tanta feminazi.\\}}}

{\small{\textit{-Oye, oye, en primera dudo que todos los de la Facultad de Filosofia sean Feminazis...\\}}}

{\small{\textit{-Es obvioooo que no son todos, wey, pero, nmms, debemos dejar de hablar, ya hay que mandar a este hijo de su mauser a la V, que en 15 es nuestra clase, y está hasta el otro lado de la facultad.\\}}}

{\small{\textit{-'Ta weno, ¿pero quién le dice? Que parece muy inmerso contandonos algo sobre que libra estará en cuarto creciente de sol o alguna pendejada así.\\}}}

{\small{\textit{-Pues tú dile, que tú eres el que lo saludo sin querer.\\}}}

{\small{\textit{-¡¡Pero a ti se te ocurrió la idea de batearlo!!\\}}}

{\small{\textit{-¡¡No te hagas!! Tenías las mismas ganas que yo de irnos a la V cuando estaba a punto de decirtelo yo con mi mirada.\\}}}

{\small{\textit{Y así termina esa triste historia, a la vez que finalizo con contarles que como a mi me gustan mucho los idiomas, y me gusta ser muy específico en ocasiones, pues por eso elegí esa palabra como título, ya que en parte me identifico con esa palabra en el aspecto de ser innecesariamente particular, o sea, {\large{\textbf{siento que me representa}}}. Quízas algo tonto, ¿pero desde cuándo los sentimientos siguen a la razón?}}}

\section{\Huge{\textbf{Sección de puntos extra}}}

    \subsection{\huge{Pregunta 1}}

{\textit{\textbf{Christian Weston Charles, alias Chris Chan, es conocido como el ser humano mejor documentado de toda la historia, incluso mejor que reyes, emperadores y profetas, desde el día de su nacimiento hasta la actualidad detalle por detalle en múltiples idiomas. Hay articulos de cientos y cientos de páginas sobre él, expresados ya sea en blogs, podcast, vídeos, películas, memes, animaciones, etc Y la razón de todo ello es tán largar que es mejor no molestarme en escribirla (no es que no la conozca, obvio Bv).\\}}}

{\color{red} Los ángeles de la muerte, o también llamados adeptos a Astartes, de Warhammer 40k, son los guerreros más poderosos de la ficción. Ideados por la mente del dios emperador de la humanidad, un ser de poder divino y absoluto. Cada Astarte ha sido modificado geneticamente para superar en todo sentido a lo humano: pueden luchar días enteros sin sentir fátiga, sobrevivir al vacío del espacio, cargar pesos imposibles, podrían reventar uno de sus corazones y seguir luchando al 100\% de sus habilidades hasta que la herida sane en cuestión de minutos.} {\color{blue}Sicuadoctrinados para ser incapaces de sentir miedo alguno, y encima portando armaduras que aumentan sus ya de por sí habilidades sobre humanas, y armas de destrucción masiva. Podrían acabar con cualquier Spartan o Jedi en minutos, y el Doom Slayer la tendría díficil sólo con uno.}

\end{document}
